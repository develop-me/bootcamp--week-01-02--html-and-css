Accessibility is about making something available for everyone. Not excluding or catering for a specific subset.

\subsubsection{Who}

- Sight
- Hearing
- Learning difficulties
- Mobility difficulties
- Colour Blindness
- YOU (imagine if you broke your wrist)


\subsubsection{Why}

- Users == Money/Revenue
- Readable code == happy developers (more productive == money!)
- Legal requirement (Don’t get sued == money!)

\href{http://www.lflegal.com/2017/06/winn-dixie/}{http://www.lflegal.com/2017/06/winn-dixie/}

\subsubsection{How}

- Write good code - use the right element for the right content
- Check design - colours, fonts, interface
- Test: Wave Accessibility Tool [http://wave.webaim.org/](http://wave.webaim.org/)


\subsubsection{Consider}

- Video captioning
- Big enough buttons
- Cluttered/busy/animated pages
- Lots of text (20ish words per line)
- Font size
- Not just using colour as feedback
- Colour contrast
- Zoomable
- Images of text

\subsection{Aria attributes}

In case you use elements that aren't for what they're suppose to be for

For when you want to give users more information about those elements \& modules

[https://www.w3.org/TR/html-aria/](https://www.w3.org/TR/html-aria/)

\subsubsection{Screen readers use aria}

```html
<button aria-label="Close" onclick="myDialog.close()">X</button>
```

\href{http://www.heydonworks.com/article/aria-label-is-a-xenophobe}{Further reading}

