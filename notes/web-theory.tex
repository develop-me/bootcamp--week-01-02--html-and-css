% setup the document
\documentclass[b5paper,openany]{book}

% setup variables ------------------------------

% your name
\newcommand\instructor{Ruth John}

% the week number
\newcommand\weekno{1}

% week main title
\newcommand\maintitle{Web Theory}

% week subtitle
\newcommand\subtitle{General information \& guides about building things for the web.}

% link to notes - relative to github.com/develop-me/
\newcommand\github{bootcamp--week--01-02--html-and-css/blob/master/notes/}

% setup functions, styling, etc.
\input{../../latex-templates/template.tex}

% the structure of the document
\begin{document}

% render the title page (uses variables above)
\tp

\quotepage{A flower does not think about competing with the flower next to it. It just blooms.}{Anonymous}

\tableofcontents

\input{../../latex-templates/preface.tex}

\chapter{General}
When you're viewing a website on the internet you're asking for a file via the url. A \textit{server} sends you this file in response to this request, which the \textit{browser} then interprets or renders for you.

Programming for the server is generally referred to as \textit{backend}, and programming for the browser or client is usually referred to as \textit{frontend}.

Here are some languages you might hear of, this is not a definitive list.

\subsubsection{Backend Languages}

\begin{itemize}
    \item Javascript
    \item PHP
    \item Python
    \item Ruby
    \item Java
    \item C / C\# / C++
    \item .net
\end{itemize}

\subsubsection{Frontend Languages}

\begin{itemize}
    \item HTML
    \item CSS
    \item Javascript
\end{itemize}

These define roles you will hear of in companies. Here is a list, again this is not definitive:

\subsubsection{Job Roles}

\begin{itemize}
    \item UX
    \item Designer / UI
    \item Front End Developers (Front End Designers, UI Developers)
    \item Back End Developers
    \item System Administrators
    \item Product Owners
    \item Project Managers
    \item Clients
    \item Testers
\end{itemize}



\chapter{Assistive Techonologies}
\input{web-theory/02_assistive/assistive.tex}

\chapter{Devtools}
\section{DevTools}

DevTools are a developer’s best friend. Being able to inspect and debug code is a \textbf{neccessary} skill when it comes to web development.
\\

\img{800px}{tools.png}{1em}{Chrome DevTools}

To open DevTools you can either right click in the browser and click 'Inspect Element', or you can use a shortcut - \texttt{CMD + OPTION + I (Chrome and Firefox), this will vary between browsers. You should ALWAYS have your DevTools open when coding.
\\

Essential Tabs for Front-end development in Chrome:
\\

\begin{tabs}
    \item Elements - most important tab for this course
    \item Network
    \item Audits
\end{tabs}

\subsection{Debugging}

The Elements tab, alongside the Inspector and Device testing tools are essential when marking up and styling web pages. They will give you:
\\

\begin{tools}
    \item A built-in text editor, with immediate feedback in the browser
    \item Insights into the CSS cascade
    \item More visual tooling when working with the Box Model, CSS Grid etc
    \item Multiple screen widths, orientation and constraints to test against
\end{tools}

This is not an exhaustive list.
\\

\subsection{Auditing}

The Network and Audit tabs evaluate your sites performance in terms of: 
\\

\begin{metrics}
    \item Accessibility
    \item Performance
    \item SEO
    \item Best Practices
\end{metrics}

This can be useful when developing your site for the first time and when looking at optimisations.
\\

\chapter{Requirement Gathering}
\subsubsection{Receiving a Brief}

Ask the following questions

\begin{itemize}
    \item Who are your users?
    \item What are they trying to do?
    \item With which technology?
    \item What are the business aims?
\end{itemize}

From that we can determine

\begin{itemize}
    \item What technologies are we going to use?
    \item Users need
    \item Does client need to be able to support it in-house?
    \item What other systems does it need to work with?
\end{itemize}

\begin{infobox}{Assumptions}
    Remember you are not your user. We have to make assumptions, but we make sure we test those assumptions and modify accordingly
\end{infobox}



\chapter{Sitemapping}
Site maps used during the planning of a Web site by its designers.

Human-visible listings, typically hierarchical, of the pages on a site.

\textit{cite: wikipedia}

\subsubsection{Sitemapping tools}

\begin{itemize}
    \item Create a list in a wordprocessor
    \item \href{https://www.gloomaps.com/}{https://www.gloomaps.com/}
    \item \href{https://bubbl.us/}{https://bubbl.us/}
\end{itemize}



\chapter{Scamping}
A quick sketch with a pen and paper

As you're discussing the site, just sketch it out

You can throw it out and start again easily

Think about:

\begin{itemize}
    \item Content hierarchy on the page
    \item Layout + UI
    \item User journeys: How to get around
    \item Responsive behavior
\end{itemize}

\begin{infobox}{Scamps \& Scamping}
    The term \textit{scamping} comes from films. Storyboards go through sketch out versions before the final storyboard is produced. This process is called scamping.
\end{infobox}


\chapter{Wireframing}
A wirefrmae is a representation of what is there and how it works.

Crucially, a way of thinking about the content and behaviour \textbf{before visual design is considered.}

That’s why wireframes will look \textbf{bland and grey.}

We want to \textbf{focus on the function} of the site before we start thinking about style.

It will be given to:

\begin{itemize}
    \item clients.
    \item designers.
    \item developers \& sys admins.
\end{itemize}

\begin{infobox}{Educate the client}
    Clients can sometimes be unfamiliar with digital processes. When sending a wireframe link to a client you should include an explanation of what a wireframe is for, what feedback is useful, and what isn’t.
    Clients will circulate a wireframe link within the organisation, often without your helpful explanation of what a wireframe is for so add a note to the wireframe saying what it is.
\end{infobox}

\subsubsection{Wireframe Software}

\begin{itemize}
    \item Balsamiq: \href{https://balsamiq.com/products/}{https://balsamiq.com/products/}
    \item Moqups: \href{https://moqups.com/}{https://moqups.com/}
    \item UXPin: \href{https://www.uxpin.com/}{https://www.uxpin.com/}
    \item AdobeXD: \href{https://www.adobe.com/uk/products/xd.html}{https://www.adobe.com/uk/products/xd.html}
    \item \href{List on Creative Bloc here}{https://www.creativebloq.com/wireframes/top-wireframing-tools-11121302}
    \item Any design software too
\end{itemize}

\begin{infobox}{Prototypes}
    Wireframing is good but it doesn't give you interactivity. You can build a prototype to show that however. CSS Frameworks (see css.pdf) can be useful for this.
\end{infobox}



\chapter{Design Handover}

\subsubsection{What do you want to ask the designer}

\begin{itemize}
    \item Artwork file
    \item Responsive designs
    \item Fonts
    \item Other assets (images, icons) can export from artwork yourself
    \item Media (videos, audio)
    \item Animations, interactivity (hovers)
\end{itemize}


\subsubsection{What to expect}

Lots of designs

\begin{itemize}
    \item Home
    \item Standard page: About
    \item Contact page
    \item Article list page: news
    \item Individual article page
\end{itemize}


\subsubsection{Different program files}

\begin{itemize}
    \item Photoshop (PSD)
    \item InDesign (INDD)
    \item Possibly Illustrator
    \item Sketch (not Adobe, mac only)
    \item Figma
\end{itemize}


%Programme files: PS Will have layers to turn on and off access different elements separately. ID Vector files, layers, copy text, fonts. May include source images (links). Sketch: Super good at website & app. Vector based. Figma: We'll see

\subsubsection{Flat files ok...}

\begin{itemize}
    \item \textbf{PDFs} Good, vector-based, usually not lossy compression, often can copy text.
    \item \textbf{PNGs} Lossless.
    \item \textbf{JPGs} Lossy compression, difficult to get clean image from. Potentially different ‘colourspace’.
\end{itemize}




\chapter{Design Breakdown}
\subsection{Figma}


\subsubsection{Pages \& Layers}

Make sure sizes seem ok, have responsive designs

\subsubsection{Tools \& Menus at top}

Use select tool

\subsubsection{Select a layer}

See information on right. NB check sub-layers. Things are grouped.

\subsubsection{Export panel \& different image types}

Select layer and see bottom right

\subsection{Different image types}

\subsubsection{Raster (Bitmap)}

\begin{itemize}
    \item \textbf{png} Has transparency, good compression, can be big
    \item \textbf{jpg} No transparency but smaller than png
    \item \textbf{tiff} No place on the web really
    \item \textbf{gif} Transparency, animation, small file size, incredibly big file size
\end{itemize}

\subsubsection{Vector based}

\begin{itemize}
    \item \textbf{svg} Scalable (and actually code!)
\end{itemize}

\begin{infobox}{Blueprint}
    A blueprint is your own sketch of the site. It should include any information you might need in coding, such as fonts, colours and spacings
\end{infobox}



\chapter{Accessibility}
Accessibility is about making something available for everyone. Not excluding or catering for a specific subset.

\subsubsection{Who}

- Sight
- Hearing
- Learning difficulties
- Mobility difficulties
- Colour Blindness
- YOU (imagine if you broke your wrist)


\subsubsection{Why}

- Users == Money/Revenue
- Readable code == happy developers (more productive == money!)
- Legal requirement (Don’t get sued == money!)

\href{http://www.lflegal.com/2017/06/winn-dixie/}{http://www.lflegal.com/2017/06/winn-dixie/}

\subsubsection{How}

- Write good code - use the right element for the right content
- Check design - colours, fonts, interface
- Test: Wave Accessibility Tool [http://wave.webaim.org/](http://wave.webaim.org/)


\subsubsection{Consider}

- Video captioning
- Big enough buttons
- Cluttered/busy/animated pages
- Lots of text (20ish words per line)
- Font size
- Not just using colour as feedback
- Colour contrast
- Zoomable
- Images of text

\subsection{Aria attributes}

In case you use elements that aren't for what they're suppose to be for

For when you want to give users more information about those elements \& modules

[https://www.w3.org/TR/html-aria/](https://www.w3.org/TR/html-aria/)

\subsubsection{Screen readers use aria}

```html
<button aria-label="Close" onclick="myDialog.close()">X</button>
```

\href{http://www.heydonworks.com/article/aria-label-is-a-xenophobe}{Further reading}



\chapter{Upload to a server}
\input{web-theory/11_upload/upload.tex}

\chapter{Testing your website}
\section{Additional Resources}

\begin{itemize}[leftmargin=*]
    \item \href{https://web.dev/measure/}{Measure}: See how well your website performs. Then, get tips to improve your user experience
\end{itemize}


\nchapter{Glossary}
\begin{itemize}[leftmargin=*]
    \item
        \textbf{Selector}:
        The part of the CSS block which represents which HTML element(s) to affect
    \item
        \textbf{Property}:
        The style you want to change. Written before the colon.
    \item
        \textbf{Value}:
        What you want to set the \textit{Property} to, written after the colon.
    \item
        \textbf{Webfont}:
        Term to describe the use of custom fonts within websites
    \item
        \textbf{CDN}:
        Content Delivery Network. Geographically distributed group of servers which work together to provide fast delivery of assets
\end{itemize}


\input{../../latex-templates/colophon.tex}

\end{document}