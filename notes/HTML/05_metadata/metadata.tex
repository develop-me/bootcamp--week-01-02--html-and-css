\begin{minted}{html}
<head>
    <!-- character set -->
    <meta charset="utf-8">

    <!-- use ie edge if available -->
    <meta http-equiv="x-ua-compatible" content="ie=edge">

    <!-- ignore retina pixels -->
    <meta name="viewport" content="width=device-width, initial-scale=1">

    <!-- title and description -->
    <title></title>
    <meta name="description" content="">

    <!-- icons -->
    <link rel="apple-touch-icon" href="icon.png">
    <!-- Place favicon.ico in the root directory -->

    <!-- CSS files -->
    <link rel="stylesheet" href="css/main.css">

    <!-- JS files, might not be here -->
    <script src="js/vendor/modernizr-2.8.3.min.js"></script>
</head>
\end{minted}

\section{HTML5 Boilerplate}

Here is a \href{link to the index.html for HTML5 Boilerplate}{https://github.com/h5bp/html5-boilerplate/blob/master/src/index.html} which is maintained to the latest standard.

\section{Character sets}

The content in our web pages is composed of a sequence of characters. Those characters are stored in a computer as a sequence of bytes, which are numeric values.
\\

The way that the sequence of bytes is converted to characters depends on which key was used to encode the text and with a web page, the key we are referring to is the charcter set. A HTML document requires a character set to be defined, so that the browser understands how to parse the document.
\\

The recommended character set for web pages at time of writing is \href{https://www.w3.org/International/questions/qa-choosing-encodings#useunicode}{UTF-8}. UTF-8 is extensive enough to provide multi-lingual support for web pages and provides backwards compatibility to the more limited character set, ASCII.
\\

We define the character set for our document like so:

\begin{minted}{html}
<head>
    <meta charset="utf-8">
</head>
\end{minted}

\section{Link and Script tags}

Used to add resources to our web page which enhance the user experience. Resources can either be files sourced locally or external resources such as a CSS or JS framework.

\subsection{Link tag}

Belong in the head of the document and are most commonly used to load a CSS file.

\begin{minted}{html}
<head>
    <link rel="stylesheet" href="https://stackpath.bootstrapcdn.com/bootstrap/4.3.1/css/bootstrap.min.css" integrity="sha384-ggOyR0iXCbMQv3Xipma34MD+dH/1fQ784/j6cY/iJTQUOhcWr7x9JvoRxT2MZw1T" crossorigin="anonymous">

    <link rel="stylesheet" href="css/main.css"
</head>
\end{minted}

Link tags can also be used to load icons.

\begin{minted}{html}
<head>
    <link rel="icon" href="favicon.ico">

    <link rel="apple-touch-icon-precomposed" sizes="114x114"
      href="apple-icon-114.png" type="image/png">
</head>
\end{minted}

The most commonly used attributes are:

\begin{itemize}[leftmargin=*]
    \item href - path to resource
    \item rel - relationship between resource and our HTML document
    \item media - accepts a CSS media query
\end{itemize}

\subsection{Script tag}

Used to load a JavaScript file or to write 'inline' JavaScript within the HTML file itself.
\\

Can also be used with other languages such as WebGL or JSON.

\begin{minted}{html}
<script src="https://code.jquery.com/jquery-3.3.1.slim.min.js" integrity="sha384-q8i/X+965DzO0rT7abK41JStQIAqVgRVzpbzo5smXKp4YfRvH+8abtTE1Pi6jizo" crossorigin="anonymous"></script>

<script src="js/main.js"></script>

<script>
    console.log('Hello World!');
</script>
\end{minted}

Script tags are typically placed at the bottom of the body but can also be placed in the head. When placing script tags in the head of document it is recommended to use async defer attributes which prevent page load performance issues.

\section{Additional Resources}

\begin{itemize}[leftmargin=*]
    \item \href{https://developer.mozilla.org/en-US/docs/Web/HTML/Element/link}{MDN - the <link> tag}
    \item \href{https://developer.mozilla.org/en-US/docs/Web/HTML/Element/script}{MDN - the <script> tag}
    \item \href{https://stackoverflow.com/questions/6771258/what-does-meta-http-equiv-x-ua-compatible-content-ie-edge-do}{StackOverflow post - IE Edge meta tag}
\end{itemize}