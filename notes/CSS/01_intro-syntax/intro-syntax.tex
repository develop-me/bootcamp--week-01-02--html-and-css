\subsection{Intro \& Syntax}

CSS stands for Cascading Style Sheets. It's a specific language created to style HTML.

You style HTML elements by specifying the element and adding property/value pairs to define how it will look and where it will be placed.


\begin{minted}{css}
    body {
        background-color: red;
    }

    /* called */
    selector {
        property: value;
    }
\end{minted}

\begin{infobox}{Syntax}
    Every character is important. The curly braces, the colons and semicolons. Computers can't read CSS properly without these things in the correct place.
\end{infobox}

\subsection{Selectors}

There are various ways to \textit{select} the html element you want to write styles for in CSS. Below are the most common, however check out the Complex Selectors section for more.

\subsubsection{Element}

You can write the element itself

\begin{minted}{css}
    p {
        font-size: 1.2rem;
    }
\end{minted}

\subsubsection{Class}

There is a \texttt{class} attribute in HTML. We can add this attribute to elements and use it as a selector in out CSS. We have to remember to add a fullstop (period) at the beginning.

\begin{minted}{css}
    .myclass {
        width: 50%;
    }
\end{minted}

\subsubsection{id}

There is also an \texttt{id} attribute in HTML and we can use that in much the same way as a class. However we add a hash at the beginning instead.

\begin{minted}{css}
    #myid {
        padding: 20px;
    }
\end{minted}


\subsection{Where do we write CSS?}

Where we include CSS is important.

\subsubsection{inline}

We can add it as an attribute to an element

\begin{minted}{html}
<p style="font-size:1.2rem;">Some text here</p>
\end{minted}

\subsubsection{style element}

We can put it inside a \texttt{style} element within our HTML

\begin{minted}{html}
<style>
    .myclass {
        width: 50%;
    }
</style>
\end{minted}

\subsubsection{CSS file}

We can create a CSS file and write CSS in that. We link to it in the \texttt{head} of our html document.

\begin{minted}{html}
<link rel="stylesheet" href="css/main.css">
\end{minted}


\begin{infobox}{Where is best?}
    All three places have their advantages and drawbacks. The most common is to create a separate CSS file. This keeps our CSS together and means we can affect more than one HTML page at a time, by including the same CSS file in them.
\end{infobox}

\subsection{Whitespace}

Whitespace matters when you write values in CSS, the part after the colon. But it doesn't affect anything else. However you should structure your CSS like the examples as this is industry standard.

\subsection{Comments}

We write a CSS comment like so:

\begin{minted}{css}
/* This is a CSS comment */
\end{minted}

