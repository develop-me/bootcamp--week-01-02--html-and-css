When a website loads it needs to show the same fonts as the design.

\subsection{Changing your site's font}

The font-family property is often used on the body so that it's font is inherited by all of the elements throughout a web page.

\begin{minted}{css}
    body {
        font-family: 'Helvetica';
    }
\end{minted}

\subsubsection{Font Stacks}

Providing a font stack to the font-family property is considered best practice. Font stacks, as shown below, allow you to provide fallback fonts in the event that your primary font is unavailable. Your fonts can become unavailable if the font file has been moved or deleted or the font's server is down.

\begin{minted}{css}
    body {
        font-family: 'Helvetica', 'Arial', sans-serif;
    }
\end{minted}

We all have different computers with different fonts on, these are called \textit{system fonts} and although there are common fonts, it's highly unlikely all users will have the custom font from the design.

We need to make sure the font used in the design loads with the website. We need to serve the font files.

\begin{infobox}{Finding font files}
    The website designer should provide you with any font files they used in the design. You can google if you know the font name, or there is \href{https://www.myfonts.com/WhatTheFont}{this website (also an app) called What The Font}, which allows you to upload an image of a font and find it for yourself.
\end{infobox}

\subsection{Loading Fonts}

\subsubsection{Yourself}

\begin{minted}{css}
    @font-face {
        font-family: 'Raleway';
        font-weight: normal;
        src: url("fonts/raleway.otf") format("opentype");
    }
\end{minted}

Remember if the font has different styles to include them under the same name with the style specified (ie \texttt{font-weight: bold;}).

\subsubsection{Webfont service}

\begin{minted}{html}
<link href="https://fonts.googleapis.com/css?family=Raleway" rel="stylesheet">
\end{minted}

\subsection{Font Styling}

If a property starts with \texttt{font} by and large it affects children as well.

Common:

\begin{minted}{css}
    body {
        font-family: 'Raleway', "Arial", sans-serif;
        font-size: 1em;
        color: #666;
        text-align: right;

        line-height: 1.6;
        letter-spacing: 1px;
        text-decoration: underline;
        text-transform: uppercase;
        text-shadow: 1px 1px 1px black;
    }
\end{minted}

Worth noting:

\begin{minted}{css}
    body {
        text-indent: 2em;
        text-overflow: fade(10px);

        word-spacing: 5px;

        word-break: break-all;
        overflow-wrap: break-word;
        white-space: nowrap;
    }
\end{minted}

\subsection{Further Reading}

\href{https://css-tricks.com/guides/opentype-variable-fonts/}{CSS Tricks articles on variable fonts}

\href{https://css-tricks.com/fout-foit-foft/}{Loading fonts performantly}

\href{https://css-tricks.com/snippets/css/system-font-stack/}{System fonts}
