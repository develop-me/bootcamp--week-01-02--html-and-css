\subsubsection{Every element on a webpage is basically a rectangle.}

This is it's box - when we talk about the box model, we're talking about this box.
\\
Pretty much all the layout properties affect this box, the size of it and where it is positioned.
\\
Let's start with the basics.

\subsubsection{Every HTML element has a default set of styles}

You can check them in devtools

\subsection{Every element has a display property}

One such style is the \texttt{display} property
\\
Common default \texttt{display} values are \texttt{block} \& \texttt{inline}
\\
Blocks are boxes which take up all the width of the parent. Inline elements sit inline next to each other.
\\
Check out media, lists and tables for some others.
\\
The display property changes how the element behaves

\subsubsection{Common display values}

\begin{itemize}
    \item \texttt{block} box
    \item \texttt{inline} sits inline
    \item \texttt{inline-block} box that sits inlne (padding and margin work as expected)
\end{itemize}


\subsection{Layout properties}

\begin{itemize}
    \item \texttt{padding} Unit, can use shorthand or add side specifically (eg padding-top)
    \item \texttt{margin} Unit, can use shorthand or add side specifically (eg margin-bottom)
    \item \texttt{width} Unit or auto keyword
    \item \texttt{height} Unit or auto keyword
    \item both \texttt{min-} \& \texttt{max-} on width and height. Can't go smaller or bigger than
    \item \texttt{box-sizing} default is content-box which means dimensions affect content. Padding \& border is added. border-box include padding and border into dimensions.
\end{itemize}

\begin{infobox}{No heights needed}
    When working with content it's always a good idea not to specify height on elements. Content gets changed, browsers get smaller. You want to allow the content to flow down the page.
\end{infobox}

